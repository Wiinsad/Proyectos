\documentclass[a4paper, table,12pt,xcdraw]{article} %formato de la plantila que usamos

\usepackage[utf8]{inputenc}
\usepackage[spanish]{babel}
%\usepackage{graphicx} %Para incluir imagenes
\usepackage[pages=some]{background}
\usepackage[margin=2cm, top=2cm, includefoot]{geometry}
\usepackage{xcolor}
\usepackage[most]{tcolorbox}
\usepackage{fancyhdr} %definir el estilo de la pagina
\usepackage[hidelinks]{hyperref}
\usepackage{parskip}

% Declaracion de variables
\newcommand{\fondo}{Images/fondo3.png}
\newcommand{\fondoa}{Images/fondo2.png}
\newcommand{\logoUNI}{Images/UanlLogo.png}
\newcommand{\logoFACU}{Images/FcfmLogo.png}
\newcommand{\materia}{Cryptografia Aplicada}
\newcommand{\teacher}{Romeo Alfonzo Sanchez Lopez}
\newcommand{\semestre}{Semestre Febrero-Junio 2021}

% Declaraciones de Colores
\definecolor{machinecolor}{HTML}{1e56b8} %color text maquina
\color{white}

% configuracion del background

\backgroundsetup{
	placement=center,
	angle=0,
	scale=1,
	contents={\includegraphics{\fondo}},
	opacity=1
}

% Adicionales
\addto\captionsspanish{\renewcommand{\contentsname}{Index}}

% Comienzo del documento

\begin{document}
	\begin{titlepage}
	\thispagestyle{empty}
	\BgThispage
	\centering
	\renewcommand{\headrulewidth}{0pt} %linea del head

	\fancyhead[L]{\vspace{1.5cm}\includegraphics[height=3.85cm]{\logoUNI}}\fancyhead[R]{\includegraphics[height=3.85cm]{\logoFACU}}

	\textcolor{black}{.}
	\par\vspace{5.5cm}
	{\huge\scshape\textbf{UNIVERSIDAD AUTÓNOMA DE NUEVO LEÓN}}

	\par\vspace{2.5cm}
	{\LARGE\bfseries{\materia}}

	\par\vspace{1.5cm}
	{\large\bfseries{Proyecto Integrador - The Joker Birthday Card}}
	\par\vspace{1.5cm}

	{\Large\bfseries{\semestre}}


	\par\vspace{4cm}

	\thispagestyle{fancy}

	\begin{tcolorbox}[colback=white!5!white,colframe=black!75!black]
		.
		\par
		\Large{\textbf{Developed by:}\hspace{1cm} Samuel Jair Gonzalez Castillo}
		\par
		\Large{\textbf{Enrollment:}\hspace{1.7cm}   1813616}
		\par
		\Large{\textbf{Teacher Name:}\hspace{0.73cm} \teacher}
		\par
		.
	\end{tcolorbox}
	\end{titlepage}
%-----------------------------------------------------------------------------------------
	\newpage
	\BgThispage
	\clearpage
	\tableofcontents
	\clearpage

%-----------------------------------------------------------------------------------------
% Background 2

\backgroundsetup{
	placement=center,
	angle=0,
	scale=1,
	contents={\includegraphics{\fondoa}},
	opacity=1
}
%-------------------------------------------------------------------------------------------------
	\newpage
	\BgThispage

	\section{Introduction}
	In this project I developed a program whose objective was to encrypt a message that is taken randomly from a file and encrypt that same message together with a key in RSA.
In order to decrypt the content it was necessary for the user to solve a small challenge set in the program to allow him to see the cipher text in clear text.

	\section{Objetive}

	The objective of this integrative project is the creation of a program which encrypts information in AES and RSA simultaneously.

	\section{Scenario}
	The Joker wants to send birthday cardsto anyone in Gotham City. Obviously, it will have a funny component(he is the Joker, right?):each cardwill be \textbf{encrypted} with a \textbf{unique} symmetric keywhich will be also sent to the card’s recipient in a differentmessage (an envelope)so they can decrypt it and be happy.So, where is the funny part? Well, toanyone can decrypt their card, they need to answer a simple question. If the answer is correct, the symmetric key can be revealedand thus the message can be decrypted; else, the card will keepencrypted foreverand the fun will be lost.


	\section{Heuristics}
	First at all, the things you needto keep in mind:
\par▪The CryptoJokercan be coded in any high-level programming language you wantand feel comfortablewith.
\par▪The symmetric algorithm is \textbf{AES-256}.
\par▪The asymmetric algorithm is \textbf{RSA-2048 bit}.
\par▪The AES private key is randomly generated for each birthday message. One key will be not used to decrypt two or more different messages.
\par▪The RSA public and private keys will be unique. The same key pair will be used to encrypt and decrypt all the \textbf{private AES keys}, and they are never revealed to the recipient.
\par▪RSA keys must be generated in advance with some toollike \textbf{OpenSSL} andstored in a file so the CryptoJoker can read themfrom the file when needed. Those keys will not change.
	\newpage
	\BgThispage

	\section{Funtions}

	In this section I will explain the functions used in the program.

	\subsection{Funtion main}
	This function is the one that is in charge of managing the whole program, the one that calls the functions and has the flow where the program goes when it executes.

	\begin{lstlisting}[language=python,caption=main]

	#!/usr/bin/env python3

	import encryRSA
	import decryRSA
	import encryAES
	import decryAES
	import sys
	import random
	import time

	def validar(r):
		try:
			temp=int(r)
			return True
		except:
			pass

	if __name__ == "__main__":


		print("\nJoker: ¿Te sientes afortunado hoy? Juguemos a mi juego preferido, la ruleta rusa. Si te vuelas la cabeza… ¡Ganas!\nEspera!..... Cumples años hoy? vaya vaya\nTe tengo una sorpresa....")
		num = random.randint(1, 6)
		#print(num)
		with open('cartitas.txt', 'r') as f:
			n = 0
			for i in f:
				n = n+1
				if n == num:
					fel = i

		#print(fel)
		#print(type(fel))
		mensEncJ = encryAES.aes_encry(fel)

		#Cifrando en AES

		aesMSG = mensEncJ['cipher_text']
		aesKEY = mensEncJ['key']

		time.sleep(2)
		print('\nJoker: Ten tu regalito jejeje')

		regalito = encryRSA.rsa_encry(aesKEY.encode())
		print(regalito)

		blu = True

		while blu == True:

			print("\nQuieres saber que significa? (y, n o exit)")
			res = input("R: ")
		#y = 1
		#n = 2
		#print(r)
		#print(y)
			if res == str('y'):

				print("\nEntonces contesta esta pqueña preguntitaaa....")
				n1 = random.randint(1,50)
				n2 = random.randint(1,50)

				print("\nCuanto es " + str(n1)+ "+" + str(n2)+"?")
				r2 = sys.stdin.readline()
				#print(r2)

				if validar(r2):
					print("\nMuy muy bien!!")
					regalo2 = int(input("Ingresa tu regalo:\n"))

					if validar(regalo2):
						if int(regalo2) == regalito:
							print("\nVeamos que dicee!..")
							cart = decryRSA.rsa_decry(int(regalo2))
							cartD = decryAES.aes_decry(mensEncJ,cart).decode()
							print(cartD)
						else:
							print("\nUyy casi casi")
					else:
						print("\nNop Nop Ese no es.....\n")
				else:
					print("\nCasi.....")



				break

			elif res == str('n'):

				print("\nTu te lo pierdes!!\n**BOOOM** **Explota algo**....")
				break
			elif res == str('exit'):
				print("\nTu te lo pierdes!!\n**BOOOM** **Explota algo**....")
				break
			else:

				print("\nMal maal mala respuesta, responde a lo que te digo bien:)")
				pass

	\end{lstlisting}

	\newpage
	\BgThispage
 aaaa
\end{document}
